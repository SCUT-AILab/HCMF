\documentclass[10pt,twocolumn,letterpaper]{article}

\usepackage{iccv}
\usepackage{times}
\usepackage{epsfig}
\usepackage{graphicx}
\usepackage{amsmath}
\usepackage{subfigure}
\usepackage{multirow}
\usepackage{amsthm,mathrsfs,amsfonts,dsfont}
\usepackage{algorithm}
\usepackage{algorithmic}
\renewcommand{\algorithmicrequire}{\textbf{Input:}}
\renewcommand{\algorithmicensure}{\textbf{Output:}}
\usepackage{epstopdf}
\usepackage{bm}
%\usepackage[vlined,boxed,ruled]{algorithm2e}
\usepackage{url}
\usepackage{xspace}

\newtheorem{corollary}{Corollary}
\newtheorem{definition}{Definition}
\newtheorem{theorem}{Theorem}
\newtheorem{proposition}{Proposition}
\newtheorem{lemma}{Lemma}
\newtheorem{remark}{Remark}



\newcommand\mycomment[1]{}


\def\ranksym{p}
\def\symp{r}

\def\calP{\mathcal{P}}


\def\M{\mathcal{M}}
\def\R{{\mathbb R}}
\def\bR{{\bf R}}
\def\U{{\bf U}}
\def\V{{\bf V}}
\def\diag{\mbox{diag}}
\def\bsigma{\mbox{{\boldmath $\sigma$}}}
\def\trsp{{\sf T}}
\def\I{{\bf I}}
\def\0{{\bf 0}}
\def\E{{\bf E}}
\def\G{{\bf G}}
\def\grad{{\text{grad}}}
\def\bZ{{\bf Z}}
\def\bfeta{\mbox{{\boldmath $\eta$}}}
\def\mT{{\mathcal T}}

\def\bA{{\bf A}}
\def\bB{{\bf B}}
\def\bN{{\bf N}}
\def\bM{{\bf M}}
\def\bD{{\bf D}}
\def\bI{{\bf I}}
\def\bE{{\bf E}}
\def\blambda{{\bm \lambda}}
\def\calL{{\mathcal{L}}}
\def\calC{{\mathcal{C}}}
\def\calS{{\mathcal{S}}}
\def\calF{{\mathcal{F}}}
\def\bL{{\bf L}}
\def\bO{{\bf O}}
\def\bU{{\bf U}}
\def\bV{{\bf V}}
\def\dsR{\mathds{R}}
\def\bX{{\bf X}}
\def\bx{{\bf x}}
\def\btx{{\tilde{\bf x}}}
\def\by{{\bf y}}
\def\bw{{\bf w}}
\def\btw{{\tilde{\bf w}}}
\def\tK{\tilde{K}}
\def\txi{\tilde{\xi}}
\def\tildeb{{\tilde{b}}}
\def\tphi{{\tilde{\phi}}}
\def\bz{{\bf z}}
\def\br{{\bf r}}
\def\bv{{\bf v}}
\def\bb{{\bf b}}
\def\bp{{\bf p}}
\def\bA{{\bf A}}
\def\bI{{\bf I}}

\def\bx{{\bf x}}
\def\bX{{\bf X}}
\def\by{{\bf y}}
\def\bY{{\bf Y}}
\def\bw{{\bf w}}
\def\bW{{\bf W}}
\def\balpha{{\bm \alpha}}
\def\bmu{{\bm \mu}}
\def\bK{{\bf K}}
\def\bg{{\bf g}}
\def\bp{{\bf p}}
\def\p{p}
\def\bP{{\bf P}}

\def\balpha{{\bm \alpha}}
\def\bbeta{{\bm \beta}}
\def\eg{{\emph{e.g.}}}
\def\zerocolumn{{\bf 0}}

\def\ttTP{{\tt TP}}
\def\ttFP{{\tt FP}}
\def\ttFN{{\tt FN}}

\def\st{{\text{s.t.}}}
\def\etc{\emph{etc}}
\def\wrt{\emph{w.r.t}}
\def\ie{\emph{ie}}
\def\eg{\emph{eg}}
\def\rank{{\text{rank}}}
\def\Tr{{\text{Tr}}}

\def\yanred{\textcolor{red}}
\def\yanblue{\textcolor{blue}}

% Include other packages here, before hyperref.

% If you comment hyperref and then uncomment it, you should delete
% egpaper.aux before re-running latex.  (Or just hit 'q' on the first latex
% run, let it finish, and you should be clear).
\usepackage[pagebackref=true,breaklinks=true,letterpaper=true,colorlinks,bookmarks=false]{hyperref}

% \iccvfinalcopy % *** Uncomment this line for the final submission

\def\iccvPaperID{283} % *** Enter the ICCV Paper ID here
\def\httilde{\mbox{\tt\raisebox{-.5ex}{\symbol{126}}}}

% Pages are numbered in submission mode, and unnumbered in camera-ready
\ificcvfinal\pagestyle{empty}\fi
\begin{document}

%%%%%%%%% TITLE3
\title{Supplemental Material for \\``Late Fusion via Subspace Search with Consistency Preservation''}
\author{First Author\\
Institution1\\
Institution1 address\\
{\tt\small firstauthor@i1.org}
% For a paper whose authors are all at the same institution,
% omit the following lines up until the closing ``}''.
% Additional authors and addresses can be added with ``\and'',
% just like the second author.
% To save space, use either the email address or home page, not both
\and
Second Author\\
Institution2\\
First line of institution2 address\\
{\tt\small secondauthor@i2.org}
}

\maketitle



\section{Appendix A: The Proof of Lemma 1}

\begin{lemma}
%Given that the sequence $\{ \mu_{t} \}$ is always increasing, $\mu_{0} > 0$ and $\blambda_{0} = 0$, the sequence $\{ \blambda_{t} \}$ computed by Algorithm~\ref{alg:alm_mf} is bounded.
Given that $\rho > 1$, the sequence $\{ \mu_{t} \}$ is always increasing, $\mu_{0} > 0$ and $\blambda_{0} = 0$, the sequence $\{ \blambda_{t} \}$ computed by Algorithm~1 is bounded.
\end{lemma}

\begin{proof}
  %By the optimality of $\bE_{t+1}$ in Step 2 in Algorithm~\ref{alg:alm_mf}, we have the sub-differential of $\calL$ w.r.t. $\bE$ as zeros:
  By the optimality of $\bE_{t+1}$ in Step 2 in Algorithm~1, we have the sub-differential of $\calL$ w.r.t. $\bE$ as zeros:
  {\small{
  \begin{align}
  \label{eq:derivatives_calL_E}
    0 & \in \frac{\calL(\bX_{t+1}, \bE_{t+1}, \blambda_{t}, \mu_{t})}{\bE_{t+1}} \nonumber \\
      & = \partial(|| \bE_{t+1} ||_{1,2}) + \blambda_{t} + \mu_{t} (\calP_{\Omega} (\bL - \bX_{t+1} - \bE_{t+1})) \nonumber \\
      & = \partial(|| \bE_{t+1} ||_{1,2}) + \blambda_{t+1}   \nonumber \\
    \Rightarrow & -\blambda_{t+1} \in \partial(|| \bE_{t+1} ||_{1,2})  .
  \end{align}
  }}
  \noindent
  %Consider the computation of $\bE_{t+1}$ in Eq~(\ref{eq:column_wise_soft_thresholding}), which is associated with $\bL, \bX_{t+1}$, $\blambda_{t}$ and $\mu_{t}$.
  Consider the computation of $\bE_{t+1}$ in Eq~(17), which is associated with $\bL, \bX_{t+1}$, $\blambda_{t}$ and $\mu_{t}$.
  $\bL$ is the observation and thus bounded.
  %$\bX_{t+1}$ is bounded due to the convergence guarantee of Step 1 in Algorithm~\ref{alg:alm_mf}~\cite{vandereycken2013lowrank}.
  $\bX_{t+1}$ is bounded due to the convergence guarantee of Step 1 in Algorithm~1~[42].
  $\{ \blambda_{t} \}$ is initialized by setting $\blambda_{0} = 0$ and $\{ \mu_{t} \}$ is non-decreasing.
  %Thus when updating by Algorithm~\ref{alg:alm_mf}, $\bE_{t+1}$, $\partial(\bE_{t+1})$ and $\blambda_{t}$ are all bounded accordingly as $t$ increases.
  Thus when updating by Algorithm~1, $\bE_{t+1}$, $\partial(\bE_{t+1})$ and $\blambda_{t}$ are all bounded accordingly as $t$ increases.
  This completes the proof.
\end{proof}



\section{Appendix B: The Proof of Theorem 1}

\mycomment{
\begin{theorem}\label{theorem:alm_convergence}
  Given that the sequence $\{ \blambda_{k} \}$ is bounded, any accumulation point $(\bX^*, \bE^*)$ of Algorithm~1 is a stationary point.
\end{theorem}
}

\begin{theorem}
\label{theorem:alm_convergence}
  %Suppose the sequences $\{\bX_k\}_{k=1}^{\infty}$, $\{\bE_k\}_{k=1}^{\infty}$ and $\{\blambda_k\}_{k=1}^{\infty}$ are generated by Algorithm~\ref{alg:alm_mf}.
  Suppose the sequences $\{\bX_k\}_{k=1}^{\infty}$, $\{\bE_k\}_{k=1}^{\infty}$ and $\{\blambda_k\}_{k=1}^{\infty}$ are generated by Algorithm~1.
  As $k \to \infty$, the gradients of $\calL$ w.r.t. $\bX$ and $\bE$ vanish,
  thus any accumulation point $(\bX^*, \bE^*)$ is a stationary point.
\end{theorem}


%\begin{proof}
%  The proof can be found in Appendix A.
%\end{proof}
\begin{proof}
    %Considering the $(k+1)$-th iteration of Algorithm~\ref{alg:alm_mf}, we have the following development:
    Considering the $(k+1)$-th iteration of Algorithm~1, we have the following development:
    \begin{align}
    \label{eq:proof1}
      & \calL(\bX_{k+1}, \bE_{k+1}, \blambda_{k}, \mu_{k})   \nonumber \\
      & = || \bE_{k+1} ||_{1,2} + \langle \blambda_{k}, \bL - \bX_{k+1} - \bE_{k+1} \rangle  \nonumber \\
      & \quad + \frac{\mu}{2} || \bL - \bX_{k+1} - \bE_{k+1} ||_{F}^2         \nonumber \\
      & = || \bE_{k+1} ||_{1,2} + \frac{\mu}{2} [ || \bL - \bX_{k+1} - \bE_{k+1} ||_{F}^2  \nonumber \\
      & \quad + \frac{\mu}{2} \langle \blambda_{k}, \bL - \bX_{k+1} - \bE_{k+1} \rangle + \frac{1}{\mu_{k}^2} || \blambda_k ||_F^2 ] \nonumber \\
      & \quad - \frac{1}{2\mu_k} || \blambda_k ||_F^2    \nonumber  \\
      & = || \bE_{k+1} ||_{1,2} + \frac{\mu_k}{2} || \bL - \bX_{k+1} - \bE_{k+1} + \frac{\blambda_k}{\mu_k} ||_F^2 \nonumber \\
      & \quad - \frac{1}{2\mu_k} || \blambda_k ||_F^2   \nonumber \\
      & = || \bE_{k+1} ||_{1,2} + \frac{\mu_k}{2} || \frac{1}{\mu_k} [ \mu_k (\bL - \bX_{k+1} - \bE_{k+1}) - \blambda_k ] ||_F^2  \nonumber \\
      & \quad - \frac{1}{2\mu_k} || \blambda_k ||_F^2   \nonumber \\
      & = || \bE_{k+1} ||_{1,2} + \frac{1}{2\mu_k} ( || \blambda_{k+1} || - || \blambda_{k} ||_F^2 )   .
    \end{align}
    \noindent
    Let $f^*$ denote a local optimal objective function value, which enables the following development:
    \begin{align}\label{eq:proof2}
      f^* & = \min_{\bX + \bE = \bL, \bX \in \M_{\ranksym}} || \bE ||_{1,2}   \nonumber  \\
          & = \min_{\bX + \bE = \bL, \bX \in \M_{\ranksym}} || \bE ||_{1,2} + \langle \blambda_k, \bL - \bX - \bE \rangle \nonumber \\
          & \quad + \frac{\mu_k}{2} || \bL - \bA - \bE ||_F^2   \nonumber \\
          & \geq \min_{\bE, \bX \in \M_{\ranksym}} || \bE ||_{1,2} + \langle \blambda_k, \bL - \bX - \bE \rangle \nonumber \\
          & \quad + \frac{\mu_k}{2} ||  \bL - \bX - \bE ||_F^2     \nonumber \\
          & = \min_{\bE, \bX \in \M_{\ranksym}} || \bE ||_{1,2} + \frac{1}{2\mu_k} (|| \blambda_{k+1} ||_F^2 - || \blambda_{k} ||_F^2)   .
    \end{align}
    \indent
    Regarding that $\mu_{k+1} = \rho \mu_{k}$ where $\rho > 1$, and the sequence $\{ \blambda_k \}$ is bounded (Lemma 1), we derive that
    $\frac{1}{2\mu_k} (|| \blambda_{k+1} ||_F^2 - || \blambda_{k} ||_F^2) \rightarrow 0$ when $k \rightarrow +\infty .$
    Therefore, when $k \rightarrow +\infty$, we obtain $\min_{\bE, \bA = \bU \bV} = || \bE^* ||_{1,2} \leq f^*$.
    On the other hand, by Algorithm~1, $\bL - \bX_{k+1} - \bE_{k+1} = \mu_k^{-1} (\blambda_{k+1} - \blambda_{k}) .$
    As $k \rightarrow +\infty$, we have $\bL - \bX^* - \bE^* \rightarrow 0  ,$
    which indicates that $(\bX^*, \bE^*)$ is a stationary point.
    This completes proof.
\end{proof}




\section{Appendix C: The Proof of Lemma 2}


\begin{lemma}
  %Assuming that $\U_{\symp}$, $\V_{\symp}$ and $\bM$ are computed by Algorithm~\ref{alg:compute_riemannian_grad},
  Assuming that $\U_{\symp}$, $\V_{\symp}$ and $\bM$ are computed by Algorithm~2,
  we have {\small${\textnormal{grad}} f(\bX) = \U \bM \V + \U_{\symp} \V^{\trsp} + \U \V^{\trsp}_{\symp}$}.
\end{lemma}

\begin{proof}
\begin{align}
\label{eq:proof_lemma1}
       & \grad f(\bX) \\
       & = \U \bM \V^{\trsp} + \U_{\symp} \V^{\trsp} + \U \V_{\symp}^{\trsp}  \nonumber \\
       & = \U(\U^{\trsp} \G \V) \V^{\trsp} + (\bR_{v} - \U \bM) \V^{\trsp} + \U (\bR_{u} - \V \bM^{\trsp})^{\trsp} \nonumber \\
       & = P_{\U} \G P_{V} + \G \V \V^{\trsp} - \U \bM \V^{\trsp} + \U \U^{\trsp} \G - \U \bM \V^{\trsp} \nonumber \\
       & = P_{\U} \G P_{V} + \G P_{V} - 2\U \bM \V^{\trsp} + P_{U} \G  \nonumber \\
       & = P_{\U} \G P_{V} + \G P_{V} - 2\U (\U^{\trsp} \bR_{v}) \V^{\trsp} + P_{U} \G  \nonumber \\
       & = P_{\U} \G P_{V} + \G P_{V} - 2\U (\U^{\trsp} \G \V) \V^{\trsp} + P_{U} \G  \nonumber \\
       & = P_{\U} \G P_{V} + (\bI - P_{U}) \G P_{V} + P_{U} \G (\bI - P_{V})  \nonumber \\
       & = P_{\U} \G P_{V} + P_{U}^{\perp} \G P_{V} + P_{U} \G P_{V}^{\perp}  ,
\end{align}
\end{proof}




\end{document}